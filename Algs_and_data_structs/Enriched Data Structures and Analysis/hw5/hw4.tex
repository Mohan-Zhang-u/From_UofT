\documentclass{article}
\usepackage{amssymb}
\usepackage{tikz}
\usetikzlibrary{trees}
\usepackage{listings}
\usepackage{amsmath}% use for piece wise function.

\title{Solution For Homework 1}
\author{Mohan Zhang, 1002748716, morgan.zhang@mail.utoronto.ca}

\begin{document}
{\noindent\large\it(part c)}
Define a function ComputeAFromT(T) which returns the adjacency n by n matrix A of G, by performing BFS on n by n matrix T.\\
ComputeAFromT(T) takes O($n^2$) since BFS on a n by n matrix(T) to find all entries less than 1 takes O($n^2$). And the function returns a correct adjacency matrix A of G, since every (i,j) $\in$ E has T[i,j]$<$1, and every (i,j) $\notin$ E has T[i,j]$\geq$1, so we can find all (i,j) $\in$ E to build A.\\
Define a function ComputePFromD(D) which takes an n by n distance matrix D of G and returns a n by n matrix P, by performing "mod 2" on each of D's entries.\\
ComputePFromD(D) takes O($n^2$) since perform "mod 2" on each of D's entries takes $n^2$ times. And the function returns a correct matrix P, by the definition of P.\\
\begin{lstlisting}
Def ComputeP(Dsq,T):
	A := ComputeAFromT(T);
\end{lstlisting}
%this is a general claim, where shall I put this????


\section{Question 2.3}
Claim:\\
T[i,j]$<$1 iff (i,j) $\in$ E.\\
First,if T[i,j]<1, then (i,j) $\in$ E.\\
prove by contradiction:\\
If (i,j) $\notin$ E, then i $\notin$ $\{k:(k,j) \in E\}$. So, $\forall$ k in $\{k:(k,j) \in E\}$, $D_{sq}$[i,k] $\geq$ 1. And since deg(j) =  $|\{k:(k,j) \in E\}|$, we now have $\sum_{k:(k,j) \in E}^{} D_{sq}[i,k] \geq |\{k:(k,j) \in E\}| = deg(j)$, so T[i,j] $\geq$ 1. Then, by contradiction, if T[i,j]<1, then (i,j) $\in$ E.\\
Second,if (i,j) $\in$ E, then T[i,j]<1.
proof:\\
(i,j) $\in$ E, then $\forall$ k $\in$ $\{k:(k,j) \in E\}$, since $\delta$(k,j)=1, and $\delta$(i,j)=1, we have $\delta$(i,k) $\leq$ 2,implies $D_sq$(i,k) $\leq$ 1. And, i $\in$ $\{k:(k,j) \in E\}$, means $\exists$ k $\in$ $\{k:(k,j) \in E\}$, such that $\delta$(k,i)=0. So, $\sum_{k:(k,j) \in E}^{} D_{sq}[i,k] < deg(j)$, which implies T[i,j]<1.\\
We are done.\\
When $\delta$(i,j) $\geq$ 2, let b $\in \mathbb{N}$.\\
Claim: When $\delta$(i,j) is even, assume $\delta$(i,j)=2b. When $\delta$(i,j) is odd, assume $\delta$(i,j)=2b+1. We have $b \leq T[i,j] < b+1$.
Proof:\\
$D_{sq}$[i,j]=$\lceil \delta(i,j)/2 \rceil$, so D[i,j]=b if $\delta$(i,j) is even, and so D[i,j]=b+1 if $\delta$(i,j) is odd.% is that already proven?
For all k such that (k,j)$\in$E ,$\delta(i,j)-1 \leq \delta(i,k) \leq delta(i,j)+1$ since $\delta(k,j)$=1, so \\
when $\delta$(i,j) is even,\\
\[D_{sq}[i,k]=\lceil \delta(i,k)/2 \rceil ==\begin{cases}
b&\text{$ $ $ \delta(i,k)=(\delta(i,j)-1)$ $or$ $\delta(i,j) $},\\
b+1&
\text{$\delta(i,k)=\delta(i,j)+1 $ }.
\end{cases}\]

when $\delta$(i,j) is odd,\\
\[D_{sq}[i,k]=\lceil \delta(i,k)/2 \rceil ==\begin{cases}
b&\text{$  \delta(i,k)=\delta(i,j)-1 $},\\
b+1&
\text{$ | $ $ \delta(i,k)=\delta(i,j)$ $or$ $(\delta(i,j)+1) $ }.
\end{cases}\]

 Furthermore, there exist a k in the shortest path from i to j such that $\delta(i,k)$=$\delta(i,j)-1$ (since $\delta$(i,j) $\geq$ 2), which implies $D_{sq}[i,k]$=b for such k (from above piecewise functions). By the justification above,since T[i,j] = $\frac{1}{deg(j)} \sum_{k:(k,j) \in E}^{} D_{sq}[i,k]$, 
 b $\leq$ T[i,j] (since $D_{sq}[i,k] \geq b$) and T[i,j] $<$ b+1 (since $D_{sq}[i,k] \leq b+1$ and there exist a k such that $D_{sq}[i,k]$=b). Then we get b $\leq$ T[i,j] $<$ b+1  $\textcircled{1}$.\\
When $\delta(i,j)$ is even, we have[ $\textcircled{1}$, $D_{sq}[i,k]$=b and T[i,j]=0 ] $\textcircled{2}$. \\When $\delta(i,j)$ is odd, we have [ $\textcircled{1}$, $D_{sq}[i,k]$=b+1 and T[i,j]=1 ] $\textcircled{3}$.\\
From above $\textcircled{2}$ and $\textcircled{3}$, we know that Helper\_ComputeP is correct.\\
// this function computes (returns) P[i,j] according to given 
T[i,j] and $D_{sq}$[i,k], where T[i,j] is from the given definition
, $D_{sq}$[i,k] is the entry in squared graph's distance matrix.
\begin{lstlisting}
Helper\_ComputeP(T[i,j], $D_{sq}$[i,k]):
	if (T[i,j]'s integer part == D_{sq}[i,k])
		return P[i,j]=0
	else // (T[i,j]'s integer part) +1 == D_{sq}[i,k]
		return P[i,j]=1
\end{lstlisting}
It runs in linear time.
So we can construct function ComputeP($D_{sq}$,T).
\begin{lstlisting}
ComputeP(D_{sq},T):
	for each [i,j] 
		compute P[i,j] from 
		Helper_ComputeP(T[i,j], D_{sq}[i,k])
	return P.
\end{lstlisting}
The above function is correct since Helper\_ComputeP(T[i,j], D\_{sq}[i,k]) is correct. And it runs linear time since each statement in for loop runs linear time, and the whole for loop runs O($n^2$) times.\\
\\Done.



\end{document}